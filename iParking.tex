%"runningheads" enables:
%  - page number on page 2 onwards
%  - title/authors on even/odd pages
%This is good for other readers to enable proper archiving among other papers and pointing to content.
%Even if the title page states the title, when printed and stored in a folder, when blindly opening the folder, one could hit not the title page, but an arbitrary page. Therefore, it is good to have title printed on the pages, too.
\documentclass[runningheads,a4paper]{llncs}

% use chinese
\usepackage{CJKutf8}

% use subpackage
\usepackage{subfigure}

% tikz package
\usepackage{tikz}
\usetikzlibrary{shapes.geometric, arrows}

%Even though `american`, `english` and `USenglish` are synonyms for babel package (according to https://tex.stackexchange.com/questions/12775/babel-english-american-usenglish), the llncs document class is prepared to avoid the overriding of certain names (such as "Abstract." -> "Abstract" or "Fig." -> "Figure") when using `english`, but not when using the other 2.
\usepackage[english]{babel}

%better font, similar to the default springer font
%cfr-lm is preferred over lmodern. Reasoning at http://tex.stackexchange.com/a/247543/9075
\usepackage[%
rm={oldstyle=false,proportional=true},%
sf={oldstyle=false,proportional=true},%
tt={oldstyle=false,proportional=true,variable=true},%
qt=false%
]{cfr-lm}
%
%if more space is needed, exchange cfr-lm by mathptmx
%\usepackage{mathptmx}

\usepackage{graphicx}

%extended enumerate, such as \begin{compactenum}
\usepackage{paralist}

%put figures inside a text
%\usepackage{picins}
%use
%\piccaptioninside
%\piccaption{...}
%\parpic[r]{\includegraphics ...}
%Text...

%Sorts the citations in the brackets
%It also allows \cite{refa, refb}. Otherwise, the document does not compile.
%  Error message: "White space in argument"
\usepackage{cite}

\usepackage[T1]{fontenc}

%for demonstration purposes only
\usepackage[math]{blindtext}

%for easy quotations: \enquote{text}
\usepackage{csquotes}

%enable margin kerning
\usepackage{microtype}

%tweak \url{...}
\usepackage{url}
\urlstyle{same}
%improve wrapping of URLs - hint by http://tex.stackexchange.com/a/10419/9075
\makeatletter
\g@addto@macro{\UrlBreaks}{\UrlOrds}
\makeatother
%nicer // - solution by http://tex.stackexchange.com/a/98470/9075
%DO NOT ACTIVATE -> prevents line breaks
%\makeatletter
%\def\Url@twoslashes{\mathchar`\/\@ifnextchar/{\kern-.2em}{}}
%\g@addto@macro\UrlSpecials{\do\/{\Url@twoslashes}}
%\makeatother

%diagonal lines in a table - http://tex.stackexchange.com/questions/17745/diagonal-lines-in-table-cell
%slashbox is not available in texlive (due to licensing) and also gives bad results. This, we use diagbox
%\usepackage{diagbox}

%required for pdfcomment later
\usepackage{xcolor}

% new packages BEFORE hyperref
% See also http://tex.stackexchange.com/questions/1863/which-packages-should-be-loaded-after-hyperref-instead-of-before

%enable hyperref without colors and without bookmarks
\usepackage[
%pdfauthor={},
%pdfsubject={},
%pdftitle={},
%pdfkeywords={},
bookmarks=false,
breaklinks=true,
colorlinks=true,
linkcolor=black,
citecolor=black,
urlcolor=black,
%pdfstartpage=19,
pdfpagelayout=SinglePage,
pdfstartview=Fit
]{hyperref}
%enables correct jumping to figures when referencing
\usepackage[all]{hypcap}

%enable nice comments
\usepackage{pdfcomment}
\newcommand{\commentontext}[2]{\colorbox{yellow!60}{#1}\pdfcomment[color={0.234 0.867 0.211},hoffset=-6pt,voffset=10pt,opacity=0.5]{#2}}
\newcommand{\commentatside}[1]{\pdfcomment[color={0.045 0.278 0.643},icon=Note]{#1}}

%compatibality with TODO package
\newcommand{\todo}[1]{\commentatside{#1}}

%enable \cref{...} and \Cref{...} instead of \ref: Type of reference included in the link
\usepackage[capitalise,nameinlink]{cleveref}
%Nice formats for \cref
\crefname{section}{Sect.}{Sect.}
\Crefname{section}{Section}{Sections}

\usepackage{xspace}
%\newcommand{\eg}{e.\,g.\xspace}
%\newcommand{\ie}{i.\,e.\xspace}
\newcommand{\eg}{e.\,g.,\ }
\newcommand{\ie}{i.\,e.,\ }

%introduce \powerset - hint by http://matheplanet.com/matheplanet/nuke/html/viewtopic.php?topic=136492&post_id=997377
\DeclareFontFamily{U}{MnSymbolC}{}
\DeclareSymbolFont{MnSyC}{U}{MnSymbolC}{m}{n}
\DeclareFontShape{U}{MnSymbolC}{m}{n}{
    <-6>  MnSymbolC5
   <6-7>  MnSymbolC6
   <7-8>  MnSymbolC7
   <8-9>  MnSymbolC8
   <9-10> MnSymbolC9
  <10-12> MnSymbolC10
  <12->   MnSymbolC12%
}{}
\DeclareMathSymbol{\powerset}{\mathord}{MnSyC}{180}

% correct bad hyphenation here
\hyphenation{op-tical net-works semi-conduc-tor}

\begin{document}
\begin{CJK}{UTF8}{bsmi}

%Works on MiKTeX only
%hint by http://goemonx.blogspot.de/2012/01/pdflatex-ligaturen-und-copynpaste.html
%also http://tex.stackexchange.com/questions/4397/make-ligatures-in-linux-libertine-copyable-and-searchable
%This allows a copy'n'paste of the text from the paper
\input glyphtounicode.tex
\pdfgentounicode=1

\title{iParking -- Real-Time Parking Space Monitor and Guiding System with Cloud Service}
%If Title is too long, use \titlerunning
%\titlerunning{Short Title}
\titlerunning{iParking}

%Single insitute
\author{Ching-Fei Yang \and You-Huei Ju \and Chung-Ying Hsieh \and
Chia-Ying Lin \and Meng-Hsun Tsai \and Hui-Ling Chang}
%If there are too many authors, use \authorrunning
\authorrunning{Ching-Fei Yang et al.}
\institute{Computer Science, National Cheng Kung University}

\maketitle

\begin{abstract}
By the popularization of cars, number of vehicles owned by each person
grows with passing days. However, the number of parking areas is out of
proportion. In order to satisfy the requirements of parking space and
reduce illegal parking, we propose iParking, a real-time parking space
monitoring and guiding system, in this paper. 

We lay emphasis on roadside parking. The system determines and records
empty parking spaces through cloud computing, wireless technology
between vehicles and image analysis. It tells you the nearest location
of empty parking space while drivers have requests. We expect the system
to cause attention to more people and government, and to solve relative
problems about parking space. 
\end{abstract}

\keywords{cloud computing, image recognition, parking space management, wireless technology}

%%%%%%%%%%%%%%%%%%%%%%%%%%%%%%%%%%%%%%%%%%%%%%%%%%%%%%%%%%%%%%%%%%%%%%%%%%%%%%%
\section{Introduction}\label{sec:intro}
%%%%%%%%%%%%%%%%%%%%%%%%%%%%%%%%%%%%%%%%%%%%%%%%%%%%%%%%%%%%%%%%%%%%%%%%%%%%%%%

%
\subsection{Research motivation}
%
According to statistics, the amount of vehicles increases yearly in
Taiwan. People usually create chaos near parking space while taking
plenty of time to find it. That is to say, parking problem has become
people's harassment. It is shown from statistical data in Ministry of
Transportation, Taiwan that the number of registered vehicles is
7,554,319 until December 2014 \cite{motc}; However, it is also mentioned that
the number of legal parking space is about four million in total. We can say that
parking space is in short supply with the disparity of vehicles.
Moreover, it will cause several problems such as the extremely slow
speed while finding parking space, scrambling for roads with scooters,
parking temporarily in dangerous part, or driving U-turn illegally. The
behavior will not only break the safety and regular of transportation
but also make noise and consume resource.

Take Tainan City, Taiwan for example \cite{strategyTainan}, high road trip rate gives rise
to high parking demand in many areas in Tainan. Up to now, there are
totally about 580 thousand registered cars and trucks while around
11,000 parking spaces are planned. Furthermore, due to the unreasonable
proportion of parking spaces (including roadside parking, off-street
parking, and spaces attached in building), roadside parking space is not
enough obviously. Theoretically, the number of roadside parking space is
required to reach 20 percent. Otherwise, it will result in lack of
parking space, spending too much time finding space, or etc.

It is easy to observe that some vehicles need to find parking space by
themselves while roadside parking spaces are not enough. The situation
will bring out arbitrary parking, and it is also the main reason of
illegal parking. In addition, according to Pollster online market survey
\cite{Pollster}, around sixty percent of people are bothered by parking issue. Thus,
we build this application in order to provide nearby parking space
information to drivers immediately and reduce traffic problems.

Recently, LBS (Location-Based Service) \cite{LBS} is getting noticed along with
the appearance of smartphones. LBS can apply broadly to different area
like health, job, daily life, and etc. Thus, how to use LBS to help
different users find the appropriate parking space is vital. The usage
of monitoring parking space now is to provide roadside parking space's
locations at best, but it will not tell drivers where the vacant space
is. Thus, we would like to develop a monitor and guiding system focus on
roadside parking to provide the information of nearby parking space and
help drivers park with the fastest way.


%%%%%%%%%%%%%%%%%%%%%%%%%%%%%%%%%%%%%%%%%%%%%%%%%%%%%%%%%%%%%%%%%%%%%%%%%%%%%%%
\section{Related work}\label{sec:relatedWork}
%%%%%%%%%%%%%%%%%%%%%%%%%%%%%%%%%%%%%%%%%%%%%%%%%%%%%%%%%%%%%%%%%%%%%%%%%%%%%%%

According to the target of our goal, we will discuss the topic about
``demand of parking space'', ``Transmitting and analyzing driving records'',
and ``Comparing with existing parking space monitored technique''.

%
\subsection{Demand of parking space}
%

In the statistical table of important indicators from Ministry of
Transportation, Taiwan \cite{motc}, it is pointed out that parking space is one
of the important indicators in addition to the number of vehicles.
\Cref{fig:StatisticalTable} is the comparison chart between the number of vehicles and
parking spaces from 2006 to 2014 in Taiwan.

\begin{figure}
\begin{center}
\includegraphics[width=0.8\textwidth]{Figures/Statistical_Table.png}
\end{center}
\caption{Statistical table of vehicles and parking spaces' number from
2006 to 2014}
\label{fig:StatisticalTable}
\end{figure}

It is shown from \Cref{fig:StatisticalTable} that the difference between supply and demand
of parking space is about two million. In addition, it is pointed out
from trend of line that the growth rate of vehicles and parking spaces
is closed. However, it is not simple to add parking spaces because it
involves road network planning. In this knotty situation, it becomes
vital and urgent to solve the management of parking space in order to
make good use of limited resource.

Furthermore, some recent researches verify that people would rather
spend more time finding roadside parking space than off-street parking
even if there are vacant spaces in the off-street parking lot \cite{OnStreetParking}. Take
Tainan City for example, service in parking lot does not meet drivers'
expect, such as high parking fee, mess surroundings, or etc. Hence, the
situation results in low usage rate of parking lot and shortage of
roadside parking space. Thus, to solve the problem of roadside parking
is necessary.

%
\subsection{Transmitting and analyzing driving records}
%

%
\subsubsection{Cloud storage and computing}
%

Cloud Storage is an online service which can save data on virtual server
through Internet. The service become more and more popular due to the
popularity of Internet and the increasing demand of data storage. That
is to say, simply save data in actual hard dick is getting insufficient.
Therefore, limited storage devices will bring more benefit by Internet
and storage virtualization technique.

In order to improve the efficiency of driving records, and reduce the
capacity of mobile device. We will refer to existing cloud storage
service, analyzing data through cloud server, and send the parking
information to users who have request.

%
\subsubsection{Static image streaming}
%

Streaming media is a process to compress a series of media data, send
through network section, and offer real-time media service on the
Internet. By the technique, media data are able to watch without
downloading whole media. Therefore, it is called ``streaming'' because
data in the process behaves like running water.

We can say that static image streaming is to connect images, and to
record the event over the next period of time. Under the premise that
analyzing driving records accurately, we will use static image streaming
to lessen the burden instead of sending whole driving record.

%
\subsubsection{Analyzing vacant parking space}
%

In the reference \cite{ParkingSpot}, the author has proposed solutions to detect if
parking spaces are vacant. Its technique includes Hough line detection
and Canny edge detection, and implementing by OpenCV library. The
original method has two limitations. The first one is that it can only
identify one photo at a time while the another is that only the parking
space at bottom right corner can be identified. We breakthrough them by
using static image streaming.

%
\subsection{Existing parking space monitored technique}
%

All of existing parking space monitored techniques are limited to
parking lot and only supported by sensors. For instance, intelligent
parking lot uses wireless sensor network, ZIBEE, pressure sensors
\cite{PaymentSystem}.
They update database by sensors and knowing if it is empty. The another
instance is Eco-Community plan developed by several schools
\cite{EcoCommunity}. Its
main method is using the sensitivity of the sensor, Octopus II, and
update changes to database. Therefore, the changes will tell users
information about parking spaces. However, the two techniques have a big
constraint when it comes to downtown area. It is a huge challenge to set
up sensors for all parking spaces in downtown due to the fee of building
and maintenance. That is to say, both of them are not suitable to
roadside parking in comparison with our system.

As regards other apps in the market, they are connected to nearby
parking lot, and offering real-time information. However, only few of
them mention payment information about roadside parking. In conclusion,
none of the apps in the market provide function to find roadside parking
spaces until now.

%%%%%%%%%%%%%%%%%%%%%%%%%%%%%%%%%%%%%%%%%%%%%%%%%%%%%%%%%%%%%%%%%%%%%%%%%%%%%%%
\section{Service and System Structure}\label{sec:service}
%%%%%%%%%%%%%%%%%%%%%%%%%%%%%%%%%%%%%%%%%%%%%%%%%%%%%%%%%%%%%%%%%%%%%%%%%%%%%%%

%
\subsection{Software platform}
%

We choose smartphone and related device to complete mobility, driving
recorder, and Network communication by reason of the target users,
people with mobile vehicle. We build the application base on Android,
and using Java to implement code structure and GUI design.

%
\subsection{Features and efficacy}
%

%
\subsubsection{Monitoring parking space}
%

The situation of parking space is different from area, timing, and
location. Therefore, the key point is how to monitor the specific
parking space immediately. Besides, if there are many people use this
service at the same time, it will be fairly accurate with steadily
update.

%
\subsubsection{Saving and analyzing driving recorder}
%

In the process of detecting vacant parking space's condition, it is
necessary to analyze big data and use large storage. Hence, we use cloud
service and client-server model to handle and send all the requests in
order to reduce the usage of memory, storage, and workload.

%
\subsubsection{Data transmission}
%

If the goal is to keep high accuracy and immediacy, the system will
bring out high Internet usage because it continuing transfers driving
recorder. Therefore, we will capture driving image with a fixed distance
according to speed of the vehicle. Coordinating with GPS position, it
will become image streaming instead of video. That is to say, capacity
and the time of data transmission can be saved.

%
\subsection{Structure}
%

We mainly focus on car owners. Besides, we will use our own approach to
detect driving record automatically and communicate between vehicles.
The application is built on Android, and expected to run the program on
driving recorders. We will limit to a specific road section while
testing and developing the system.

Four steps are supposed to proceed. First, determine the specific road
section, and collect data; Next, sort out the collected data, and start
to plan the structure of program; Then, begin to develop the program,
add GUI, do simulation, and test system. Finally, analyze and present
whole the research. \Cref{fig:flowDiagram} is the flow diagram of these steps.

\begin{figure}
\begin{center}
	% Define block styles
	\tikzstyle{startstop} = [rectangle, rounded corners, minimum
width=3cm, minimum height=1cm,text centered, text width=3cm, draw=black, fill=red!30]
	\tikzstyle{process} = [rectangle, minimum width=3cm, minimum
height=1cm, text centered, text width=3cm, draw=black, fill=orange!30]
	\tikzstyle{decision} = [diamond, minimum width=3cm, minimum height=1cm,
text centered, text width=3cm, draw=black, fill=green!30]
	\tikzstyle{arrow} = [thick,->,>=stealth]
	\tikzstyle{io} = [trapezium, trapezium left angle=70, trapezium
right angle=110, minimum width=2cm, minimum height=1cm, text centered,
text width=3cm,  draw=black, fill=blue!30]
	    
	\begin{tikzpicture}[node distance=2cm]
	    % Place nodes
	    \node [startstop] (init) {initialize model};
	    \node [process, below of=init] (function) {plan for each function};
	    \node [process, below of=function] (space) {provide space to store driving record};
	    \node [io, below of=space] (receive) {receive users' requests};
	    \node [process, right of=receive, xshift=2cm] (design) {design UI interface};
	    \node [process, above of=design] (system) {implement the whole system};
	    \node [decision, above of=system, yshift=1.5cm] (testing) {test client and server side separately};
	    \node [decision, right of=testing, xshift=3cm] (integrate) {integrate all structures};
		\node [startstop, below of=integrate, yshift=-1.5cm] (end) {stop};
	    % Draw edges
	    \draw [arrow] (init) -- (function);
	    \draw [arrow] (function) -- (space);
	    \draw [arrow] (space) -- (receive);
		\draw [arrow] (receive) -- (design);
		\draw [arrow] (design) -- (system);
		\draw [arrow] (system) -- (testing);
	    \draw [arrow] (testing) -- node {yes} (integrate);
	    \draw [arrow] (testing) -- node {no}(system);
	    \draw [arrow] (integrate) -- node {yes} (end);
	    \draw [arrow] (integrate) -- node {no}(system);
	\end{tikzpicture}
\end{center}
\caption{Research flow diagram}
\label{fig:flowDiagram}
\end{figure}

%
\subsubsection{Building cloud server and algorithm design}
%

The main usage of cloud server is to implement each operated function,
handle users' request, and provide storage to save driving records. The
operated function includes receiving image data from users, analyzing
images, building database, searching database, and etc. Hold time is the
first concern because of massive calculation. The other factor is the
availability of data, for example, we will not use old data and data
which has been analyzed in the same location. This algorithm will help
reduce repeated operations, furthermore, it helps us make sure that we
are analyzing real-time image in every data.

%
\subsubsection{Program structure in device and UI design}
%

We will plan our service with relation to different characters. For
example, it is necessary for users to search nearby parking space, it is
required for devices to send image information to cloud server, and it
is important for servers to analyze and collect data. As for UI design
in app, in order to realize convenient searching function and interface,
it need to be designed from the view of users.

The error of GPS measurement is about 5 to 10 meters, and we can say
that it is about 1 or 2 roadside parking spaces. In order to enhance the
accuracy, we will combine Google Maps API, and take the advantage of its
navigation and distance matrix service. Besides accessing the speed and
distance of vehicles, the API can also help send GPS location to keep
loading and operating fast in the device.

When it comes to clients, we will check whether there is anyone else
sending the same information of specific location at the same time or
not. It will not send information if anyone else is sending the data.
However, if there is not anyone else sending the data, we plan to
capture image immediately after moving a small and appropriate distance
and send to cloud server.

\begin{figure}
	\centering
		\includegraphics[width=0.8\textwidth]{Figures/flowChart.png}
		\caption{System flow chart}
		\label{fig:system}
\end{figure}


%%%%%%%%%%%%%%%%%%%%%%%%%%%%%%%%%%%%%%%%%%%%%%%%%%%%%%%%%%%%%%%%%%%%%%%%%%%%%%%
\section{Implementataion}\label{sec:implementation}
%%%%%%%%%%%%%%%%%%%%%%%%%%%%%%%%%%%%%%%%%%%%%%%%%%%%%%%%%%%%%%%%%%%%%%%%%%%%%%%

The system is divided into three parts - image recongization, cloud
server, and client's application. At first, the three parts will be implemented
seaprately. They will be combined and operate after they all make a certain proportain.

\subsection{Image recognition of roadside parking space}

Image recoginition and analysis of roadside parking space is implemented by C++ with
OpenCV library. The program will return if the image of parking space is
vacant after recieved an image.

\subsubsection{Setting ROI (Region of Interest)}

First, the program will convert the image into gray scale. The reason is
that the perspective of image will affect the degree of image
recognition. Moreover, the image will be divided into four equal parts.
Only the right-bottom part will be reserved because the spaces are
usually located in the right hand side.

\begin{figure}
	\subfigure[Gray scale image] {
		\includegraphics[scale=0.8]{Figures/grayScaleImage.png}
	}
	\hspace{0.5in}
	\subfigure[ROI image] {
		\includegraphics[scale=0.8]{Figures/ROIImage.png}
	}
	\caption{Setting ROI}
	\label{fig:ROI}
\end{figure}

\subsubsection{Sides detection and Noice reduction on Image} 

The program uses Canny edge detection in OpenCV to find each side of
parking space. Afterward it reduces noice by the way, Median Blur.
(\Cref{sideDetecting})

\begin{figure}
	\begin{minipage}{.4\textwidth}
		\centering
		\includegraphics[scale=0.7]{Figures/sideDetecting.png}
		\caption{Detect each side}
		\label{sideDetecting}
	\end{minipage}
	\hspace{0.5in}
	\begin{minipage}{.4\textwidth}
		\centering
		\includegraphics[scale=0.7]{Figures/drawParkingSpace.png}
		\caption{Draw parking space}
		\label{drawParkingSpace}
	\end{minipage}
\end{figure}

\subsubsection{Find out parking spaces}

First, find out straight line with Hough line detection, and detect
which one is the line of parking space by angle and intersection of
lines. (\Cref{drawParkingSpace})

If the appropriate space is not found, the program will return the
result of no vacant space (equals to occupied by vehicle). However, if
the space is found, it will continue to next to detect if the space is
able to use.

\subsubsection{Detect if the parking space is able to use}

If the ratio between the side of parking space and its shelter is more
than a certain number, it means that the space has been occupied and the
program will return no vacant spaces. On the contrary, it will return
there is a vacant parking space.

\begin{figure}
	\subfigure[Available parking space] {
		\includegraphics[scale=0.7]{Figures/availableParkingSpace.png}
	}
	\hspace{0.5in}
	\subfigure[Unavailable parking space] {
		\includegraphics[scale=0.7]{Figures/unavailableParkingSpace.png}
	}
	\caption{Detect parking space}
	\label{detectingParkingspace}
\end{figure}

\subsection{Cloud analyzing server}

After passing image recognition of roadside parking space testing, the
program mentioned above will be moved to cloud server. Moreover, it will
coordinate with the open data offered by government. The data will
provide the information about roadside parking space. Therefore, we are
able to know which road sections do not have spaces, prevent analyze the
images from those sections. We implement the server by nodejs action
hero framework; In addition to offering API with http, we will provide
interface for webpages in order to let users find parking spaces
directly. Non-relational database, Mongo, is also used to accerlate
access and operated speed. 

Cloud server will translate longitude and latitude into address
information while recieving GPS information and images by users. Next,
it will compare the road section with open data to confirm if the
section provides parking spaces. After successful analysis, the result
with address will update to database for other users.

\subsection{Mobile APP (client side)}

It is divided into several steps to implement, and mainly separated into
user (device) and server side. First, we use Android platform with
Android Studio and Android SDK, which are based on JAVA, to develop
client’s application.


\begin{figure}
	\subfigure[Home] {
		\includegraphics[scale=0.3]{Figures/userHome.png}
		\label{fig:userHome}
	}
	\hspace{0.1in}
	\subfigure[Record] {
		\includegraphics[scale=0.3]{Figures/userRecord.png}
		\label{fig:userRecord}
	}
	\hspace{0.1in}
	\subfigure[Setting] {
		\includegraphics[scale=0.3]{Figures/userSetting.png}
		\label{fig:userSetting}
	}
	\hspace{0.1in}
	\subfigure[Result] {
		\includegraphics[scale=0.3]{Figures/userResult.png}
		\label{fig:userResult}
	}
	\hspace{0.1in}
	\subfigure[Map] {
		\includegraphics[scale=0.3]{Figures/userMap.png}
		\label{fig:userMap}
	}
	\caption{User interface}
	\label{fig:userInterface}
\end{figure}

\Cref{fig:userInterface} is our main interface and function.
\Cref{fig:userHome} is the first screen of the system. You
can choose the function of both recording and searching.
\Cref{fig:userRecord} is recording screen. It will use device's
camera automatically, and temporarily save the records in iParking
folder in order to let users check the
record, and choose if they want to provide it to other users or delete it. After
uploading the record, it will be deleted. \Cref{fig:userSetting}
is the setting page. Users can change their sending rate (network flow)
and searching range while finding parking space.
\Cref{fig:userResult} and \Cref{fig:userMap} are both the part of searching
parking spaces. Users will know how many parking spaces nearby and
they can be navigated to the nearest space combined with Google api.

\subsection{Testing}

In the part of testing, we will initially test client and server side
separately, and then merge them with UI design. Finally, we will test
the integrated system by the following steps. First, a single vehicle.
That is to test with different speed, and make sure that static image
streaming is worked with cloud server. Next, test the efficiency of
multiple vehicles, and verify that no images will be in the same
location at the same time. The last step is to confirm that client side
can get the correct information immediately.

%%%%%%%%%%%%%%%%%%%%%%%%%%%%%%%%%%%%%%%%%%%%%%%%%%%%%%%%%%%%%%%%%%%%%%%%%%%%%%%
\section{Conclusion and Future Work}\label{sec:conclusion}
%%%%%%%%%%%%%%%%%%%%%%%%%%%%%%%%%%%%%%%%%%%%%%%%%%%%%%%%%%%%%%%%%%%%%%%%%%%%%%%

\subsection{Conclusion}

本系統之目的是希望透過我們的停車查詢系統,能提供使用者一個實用又好操作的應用程式,使其有路邊停車需求時能快速找到停車位,降低查找車位時所造成的交通問題、空氣污染問題與違規停車的機率。

目前已經有一個簡單卻完整的系統,包含了遠端伺服器、伺服器中內建的影像分析技
術,以及 Android 手機 APP 可以供使用者實際使用,對於未來,除了希望能在
iPhone 和 Windows phone
所搭載的作業系統開發相同的手機應用程式外,我們更希望能再針對影像
辨識準確率、壓力測試、車載網路這三大方向進行努力。 

\subsection{Future Work}

\subsubsection{影像辨識準確率}

目前影像辨識準確率因受環境因素影像`在夜晚、雨天,甚至天氣偏陰暗時會有偏高的錯誤率。對於天氣陰暗這點,理論上可以用不同的環境變數去克服,只是得讓系統得以事先得知當時的天氣狀況,自行判斷該用哪組環境變數。目前想到的可行方案為以當天氣象局所公布的氣象資訊作為參考依準。

至於夜晚和雨天,就得直接從演算法的部分著手,夜晚或許可以先將影像整體對比度及亮度調高,使得畫面中的停車格線和停車格中的車輛能夠明顯到足以辨識的程度,才有辦法做進一步的分析;雨天則得先去掉雨滴所造成的影像雜訊與克服影像模糊的部分。對於這兩種狀況目前仍在研讀他人論文,思考可行解決方案,相信有朝一日該系統定能克服此兩種因環境所造成的低辨識率狀況。

\subsubsection{壓力測試}

由於目前系統尚未有多人同時使用的狀況,因此其實不確定當使用者一多時系統是否仍能負擔。針對該部分,當初是希望能先將系統修改至更加完善後,再推出給親朋好友試用,一邊希望他們能夠給些回饋讓系統更加穩定,另一方面也可以做壓力負載的測試,確認現在伺服器是否能夠支持使用該系統的用戶皆能正常使用。

\subsubsection{車載網路}

車載網路是一種透過隨意網路提供車輛之間的通訊,藉由無線通訊與資料傳遞技術,串聯交通工具以及路邊交通設施,所形成的特殊的專用網路,屬於高度客制化的行動式隨意網路,且比起一般隨意網路
(Ad Hoc),車載網路擁有相對較高的速度\cite{AdHoc}。現今車載網路之應用範疇多為智慧型運輸系統 (Intelligent
Transportation System, ITS
),其九大服務項目中包含安全性應用、交通管理與環境資訊與便利生活訊息傳送幾大範圍。

希望當未來 VANET 技術在台灣發展更加成熟時,本系統可以透過車載網路,讓用戶端的車輛彼此間得以直接進行溝通,使得同一時段內,同一相近車輛群中,只會有一台車輛負責上傳行車記錄影像,避免停車位計數錯誤與影像重複分析所造成的額外負擔。


%%%%%%%%%%%%%%%%%%%%%%%%%%%%%%%%%%%%%%%%%%%%%%%%%%%%%%%%%%%%%%%%%%%%%%%%%%%%%%%
\bibliographystyle{splncs03}
\bibliography{iParking}


All links were last followed on September 11, 2016.
%%%%%%%%%%%%%%%%%%%%%%%%%%%%%%%%%%%%%%%%%%%%%%%%%%%%%%%%%%%%%%%%%%%%%%%%%%%%%%%

\end{CJK}
\end{document}
